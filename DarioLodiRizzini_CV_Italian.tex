\documentclass[11pt]{article}
\usepackage[italian]{babel}
\usepackage{fullpage}
\usepackage{lastpage}
\usepackage{times}
\usepackage{bibentry}
\usepackage{notoccite}
\nobibliography{pubblicazioni}

%\usepackage{fancyhdr}
\makeatletter
\renewcommand{\@evenfoot}%
  {\normalsize \hfil
   \slshape \footnotesize pagina {\thepage} di \pageref{LastPage}}
\renewcommand{\@oddfoot}{\@evenfoot}
\providecommand{\bibitem}[1]{}
\makeatother

\newcommand{\ITEMDATE}[1]{\item \textit{#1:}\\}
\newcommand{\BIBCITE}[1]{\cite{#1} \bibentry{#1}}
\newcommand{\BIBCITEM}[1]{\item[\cite{#1}] \bibentry{#1}}



\begin{document}

\bibliographystyle{unsrt_dlr}
\nocite{lodirizzini2015ijars, aleotti2014jirs, cigolini2014jamris, lodirizzini2009ras, 
kallasi2015oceans, oleari2015oceans,
aleotti2014iros, oleari2014ifac, lodirizzini2014ias, kallasi2014iarp,
lodirizzini2013ecmr, valeriani2013acra, oleari2013iccp, 
aleotti2012icra, lodirizzini2012icinco, calo2012icinco, 
lodirizzini2011icra, 
lodirizzini2010iros, argenti2010isr, 
lodirizzini2009iros, lodirizzini2009ecmr, lodirizzini2009icar, 
grisetti2008icra, lodirizzini2008ciras, 
lodirizzini2007ecmr, lodirizzini2007icinco, 
%
kallasi2014amra, lodirizzini2012icraworkshop, lodirizzini2010graphbot, lodirizzini2009icraworkshop, cerri2007ccmvs,
%
lodirizzini2008improved,
lodirizzini2009thesis}

%%%%%%%%%%%%%%%%%%%%%%%%%%%%%%%%%%%%%%%%%%%%%%%%%
% TITOLO E RECAPITO
%%%%%%%%%%%%%%%%%%%%%%%%%%%%%%%%%%%%%%%%%%%%%%%%%

\begin{center}
\textbf{\LARGE CURRICULUM VITAE ET STUDIORUM} \\
{\Large DI DARIO LODI RIZZINI}
\end{center}


\section{Informazioni Generali} 

Nato ad Asola (MN), il 23 maggio 1981. \\
Codice Fiscale LDRDRA81E23A470D. \\
Cittadinanza: italiana. \\
Obblighi militari: dispensato. \\
Residenza e Domicilio: via E. Fermi, 11, 46010 Commessaggio (MN). \\
Cellulate: +39 333 6161152. \\
E-mail: \verb|dario.lodirizzini@gmail.com|\\
PEC: \verb|dario.lodirizzini@pec.it|\\

\section*{Indirizzo di Lavoro}

Dipartimento di Ingegneria dell'Informazione,\\
Universit\`a degli Studi di Parma\\
viale Parco Area delle Scienze, 181A I-43124 Parma Italy\\
Tel.:+39 0521 906147, Fax: +39 0521 905723\\
E-mail: \verb|dario.lodirizzini@unipr.it|\\

% ---------------------------------------------------------
% CARRIERA SCOLASTICA
% ---------------------------------------------------------

\subsection*{Carriera Scolastica e Formazione}

\begin{itemize}

\ITEMDATE{Gennaio 2006 - Marzo 2009}
  \emph{Dottorato di Ricerca in Tecnologie dell'Informazione} 
  XXI ciclo, conseguito il 3 marzo 2009 presso l'Universit\`a degli Studi di Parma.
  Tesi di Dottorato dal titolo ``Computation and Time Constraints in Localization and Mapping Problems''.

\ITEMDATE{Ottobre 2003 - Dicembre 2005}
\emph{Laurea Specialistica in Ingegneria Informatica}
conseguita il 15 dicembre 2005 presso l'Universit\`a degli Studi di Parma, 
con valutazione 110/110 e lode.
Tesi di Laurea dal titolo ``Progettazione di una Libreria per la Localizzazione 
e Fusione Sensoriale basata su Filtri Particellari''.

\ITEMDATE{Ottobre 2000 - Settembre 2003}
\emph{Laurea in Ingegneria Informatica} (triennale) 
conseguita il 25 settembre 2003 presso l'Universit\`a degli Studi di Parma, 
con valutazione 110/110 e lode.
Tesi di Laurea dal titolo ``La Localizzazione in interni tramite una Rete Wireless 
Ethernet''.

\ITEMDATE{Luglio 2000}
\emph{Maturit\`a scientifica} conseguita
presso l'Istituto di Istruzione Tecnico-Scientifica E. Sanfelice di Viadana (MN) con votazione di 100/100.

\end{itemize}

% ---------------------------------------------------------
% POSIZIONE ATTUALE
% ---------------------------------------------------------

\subsection*{Posizione Attuale}

Dario Lodi Rizzini \`e titolare dell'Assegno di Ricerca (SSD ING-INF/05) dal tema ``Metodi probabilistici per il riconoscimento
di oggetti in compiti di manipolazione e navigazione robotica''
presso il Dipartimento di Ingegneria dell'Informazione dell'Universit\`a degli Studi di Parma.

% ---------------------------------------------------------
% ATTIVITA' DI RICERCA
% ---------------------------------------------------------

\section{Attivit\`a di Ricerca}

Dario Lodi Rizzini svolge attivit\`a di ricerca nell'ambito della Robotica occupandosi 
di problemi di localizzazione, mappatura e navigazione di robot mobili,
percezione avanzata e riconoscimento di oggetti.
Tutte le tematiche sono indagate tenendo in considerazione l'esigenza di elaborazione 
in tempo reale o in linea per l'esecuzione dei compiti da parte del sistema robotico. 
Dall'inizio del suo dottorato (2006) l'attivit\`a di ricerca di Dario Lodi Rizzini ha 
portato a 
\textbf{4 articoli su riviste internazionali}, 
\textbf{27 articoli su atti di conferenze internazionali e workshop},
\textbf{1 capitolo su libro internazionale} 
per un totale di \textbf{32 contributi scientifici}.
Ulteriori 2 articoli sono in fase di valutazione presso sedi internazionali.
 
L'attivit\`a si pu\`o riassumere in quattro temi principali.  

\subsection*{Vincoli Temporali nei problemi di Localizzazione e Mapping}

Studio dei problemi di \emph{localizzazione e mapping} (SLAM) di robot mobili con particolare 
attenzione alla presenza di vincoli temporali e all'evoluzione dell'ambiente nel tempo.
Ha esteso il \emph{Real-Time Particle Filter} (RTPF), una variante 
avanzata del localizzatore basato su filtro particellare in grado di 
garantire il rispetto di vincoli real-time, introducendo un diverso metodo 
per il calcolo di parametri critici per la convergenza del RTPF~\cite{lodirizzini2007icinco,lodirizzini2007ecmr,lodirizzini2008improved}.
Nell'attivit\`a di ricerca \`e stata impiegata l'occupancy grid map che permette di integrare 
in un'unica rappresentazione le misure ottenute da sensori di prossimit\`a come i laser scanner.
%Sono state studiate altre rappresentazioni computazionalmente meno onerose ed adatte a gestire 
%oggetti semi-statici per consentire la navigazione efficiente di AGV industriali. 

\textbf{Pubblicazioni inerenti:}
3 lavori in atti di conferenze internazionali~\cite{grisetti2008icra,lodirizzini2007icinco,lodirizzini2007ecmr},
1 capitolo di libro~\cite{lodirizzini2008improved}.

\subsection*{Metodi di Graphical SLAM}

Ha studiato la formulazione \emph{maximum likelihood} (ML) dei problemi di mapping
che permette di individuare le dipendenze tra gli elementi della mappa e di decomporre il 
problema generale in sottoproblemi computazionalmente pi\`u semplici. 
In collaborazione con il gruppo dell'Universit\`a di Freiburg ha
proposto un risolutore di mappe basato su \emph{stochastic gradient descent} adatto 
all'esecuzione in linea~\cite{grisetti2008icra,lodirizzini2009ras}. 

Le caratteristiche del paradigma ML mapping sono state sfruttate per partizionare la mappa 
in cluster indipendenti, che possono essere elaborati in parallelo~\cite{lodirizzini2009iros}, consentendo
la risoluzione della mappa in contesti distribuiti e multi-robot~\cite{lodirizzini2010iros,lodirizzini2010graphbot}.
Una diversa decomposizione del problema ML SLAM ha consentito di proporre
un algoritmo di costruzione di mappe multi-ipotesi al fine di 
fornire un metodo robusto agli errori nell'associazione di dati\cite{lodirizzini2011icra}.
Tale metodo si propone di estendere i vantaggi delle metodologie multi-ipotesi tradizionali applicate
originariamente al mapping topologico alla stima di mappe metriche. 

Tale formulazione dei problemi di SLAM si presta all'analisi formale:
Dario Lodi Rizzini ricavato una soluzione parziale in forma chiusa del ML mappping, 
che dimostra le relazioni tra le variabili del problema~\cite{lodirizzini2009ecmr,lodirizzini2009icar}.
\`E stata evidenziata la dipendenza tra le variabili 
che rappresentano l'orientamento del robot e quelle che ne definiscono posizione.

\textbf{Pubblicazioni inerenti:}
1 lavoro su rivista internazionale~\cite{lodirizzini2009ras},
7 lavori in atti di conferenze internazionali~\cite{
lodirizzini2011icra,lodirizzini2010iros,lodirizzini2010graphbot,
lodirizzini2009iros,lodirizzini2009ecmr,lodirizzini2009icar,grisetti2008icra}.

\subsection*{Percezione tridimensionale ed identificazione e riconoscimento di oggetti}

L'elaborazione di dati sensoriali ottenuti con sensori di percezione tridimensionale e 
rappresentati della struttura dati \emph{point cloud} \`e un tema che ha attirato recentemente l'interesse 
della comunit\`a scientifica. 
Dario Lodi Rizzini ha lavorato con sensori basati su diverse tecnologie, come laser scanner~\cite{aleotti2014jirs,aleotti2012icra},
range camera~\cite{aleotti2014iros,lodirizzini2014ias}, visione monoculare e stereoscopica~\cite{lodirizzini2015ijars,
oleari2014ifac,oleari2013iccp} ed in setup sperimentali differenti.

Uno dei temi di cui si \`e occupato \`e il rilevamento ed il riconoscimento di oggetti finalizzato 
alla presa e manipolazione tramite braccia robotiche. 
Una situazione particolarmente interessante \`e rappresentata da sensori montati sul robot manipolatore 
\emph{eye-in-hand} perch\`e consente di controllare il punto di osservazione. 
Nei lavori~\cite{aleotti2014jirs,aleotti2012icra} l'osservazione completa dell'oggetto di interesse
\`e ottenuta alternando fasi di osservazione invevitabilmente parziale e di manipolazione finalizzata 
all'osservazione delle parti in precedenza non osservate. 
Egli ha anche studiato il problema dell'osservazione di oggetti utilizzando sensori con campo visivo 
e range differenti, attraverso un'oculata pianificazione della posizione del sensore~\cite{aleotti2014iros}.

In~\cite{lodirizzini2014ias} ha proposto un metodo di segmentazione non supervisionata di 
point cloud ottenute con sensori di percezione 3D. 
Il metodo proposto suddivide la point cloud in voxel e caratterizza ciascun voxel con un vettore 
di feature (colore, pattern e forma). 
La somiglianza tra feature e le relazioni di prossimit\`a tra voxel sono utilizzate per classificare 
ciascun voxel con approccio \emph{Markov Random Field} e, operando sui dati etichettati, 
per rilevare potenziali oggetti. 
L'oggetto candidato viene, infine, confrontato con un database di modelli con tecniche di registration
e di associazione di keypoint come quello descritto in~\cite{oleari2013iccp}.
 
Nell'ambito della partecipazione al progetto PRIN MARIS,
Dario Lodi Rizzini ha partecipato allo sviluppo di sistemi di visione subacquea stereoscopica 
per il riconoscimento e la stima di posizione degli oggetti finalizzata alla manipolazione.
Ha contribuito alla realizzazione di sistemi di visione impermeabilizzati ed allo svolgimento
di attivit\`a sperimentale di acquisizione di dataset in un ambiente difficile come quello 
subacqueo~\cite{oleari2014ifac,oleari2015oceans}.
Inoltre, ha sviluppato metodi di segmentazione dell'immagine e di riconoscimento oggetto 
per un ambiente difficile come quello sottomarino, 
impiegando tecniche come la segmentazione basata su grafo~\cite{kallasi2015oceans} e 
la clusterizzazione basata su aree~\cite{lodirizzini2015ijars,oleari2014ifac,kallasi2014amra}.
Inoltre, ha sviluppato soluzioni di visione stereoscopica sparsa per stimare la posa 3D 
degli oggetti di interesse anche con pattern deboli ed in presenza di fenomeni di attenuazione 
e back-scattering. 
%Il sistema di visione sviluppato e le metodologie proposte sono stati valutati con attivit\`a 
%sperimentale sul campo e confrontandosi con i pochi dataset disponibili in letteratura. 

\textbf{Pubblicazioni inerenti:}
2 lavori su riviste internazionali~\cite{aleotti2014jirs,lodirizzini2015ijars},
7 lavori in atti di conferenze internazionali~\cite{aleotti2012icra,oleari2013iccp,
lodirizzini2014ias,aleotti2014iros,kallasi2015oceans,oleari2015oceans}.

\subsection*{Applicazioni Industriali e Robotica sul Campo}

Nel corso degli anni Dario Lodi Rizzini ha affrontato problemi di robotica applicata in contesti 
di carattere industriale e di robotica sul campo nell'ambito di convenzioni tra universit\`a 
ed aziende del settore dell'automazione. 
Un esempio significativo \`e la simulazione e la programmazione di macchine automatiche 
come il pallettizzatore a formazione di strato pallet~\cite{argenti2010isr,calo2012icinco}. 
Tale attivit\`a ha velocizzato notevolmente lo studio del comportamento della macchina e 
la generazione di programmi personalizzati per il cliente, portando ad un reale vantaggio 
competitivo. 
Infine, ha contribuito al miglioramente della navigazione di Automated Guided Vehicle (AGV)
a guida laser, ossia veicoli automatici impiegati nella logistica, nel trasporto di materiale
e nella gestione dei magazzini. 
In~\cite{lodirizzini2007icinco} ha proposto un metodo per mappare e rappresentare gli 
oggetti semi-statici dell'ambiente industriale, ossia elementi non presenti stabilmente 
nell'ambiente e non rilevabili tramite tracciamento del loro moto, che possono ridurre 
l'efficienza degli AGV se gestiti con le politiche di sicurezza ordinarie. 

La partecipazione a competizioni di robotica mobile, oltre a rappresentare un 
rafforzamento dell'attivit\`a didattica tramite il coinvolgimento di studenti, 
ha portato alla realizzazione di sistemi robotici completi. 
Il robot mobile realizzato per la partecipazione al Sick Robot Day 2012 ha permesso 
di approfondire i problemi di progettazione e realizzazione di sistemi robotici 
in grado di svolgere compiti in ambienti reali e non semplicemente di laboratorio~\cite{cigolini2014jamris}. 
Altre esperienze di laboratorio hanno consentito di sviluppare sistemi completi 
orientati allo svolgimento di task di esplorazione anche con percezione complessa~\cite{valeriani2013acra}.


\textbf{Pubblicazioni inerenti:}
1 lavoro su rivista internazionale~\cite{cigolini2014jamris},
3 lavori in atti di conferenze internazionali~\cite{valeriani2013acra,lodirizzini2012icinco,calo2012icinco,argenti2010isr}.


% ---------------------------------------------------------
% TUTOR
% ---------------------------------------------------------

\section*{Altre Attivit\`a Scientifiche}

\begin{itemize}
\item Dario Lodi Rizzini (oltre ad essere stato correlatore di pi\`u di 20 tesi di laurea) \`e co-tutor di 2 dottorandi: Fabio Oleari e Fabjan Kallasi, dottorandi in Tecnologie dell'Informazione presso il Dipartimento di Ingegneria dell'Informazione dell'Universit\`a degli Studi di Parma rispettivamente del XXVIII e del XXIX ciclo. 
\item Dario Lodi Rizzini \`e membro della IEEE ed iscritto alla IEEE Robotic and Automation Society (RAS) dal 2006. 
\end{itemize}

% ---------------------------------------------------------
% INCARICHI
% ---------------------------------------------------------

\section*{Assegni di Ricerca ed Incarichi}

\begin{itemize}
\ITEMDATE{Giugno 2013 - Presente} 
Titolare di \emph{Assegno di Ricerca} sul tema 
``Metodi probabilistici per il riconoscimento di oggetti in compiti di manipolazione e di navigazione robotica''
presso il Dipartimento di Ingegneria dell'Informazione dell'Universit\`a di Parma.

\ITEMDATE{Marzo 2009 - Aprile 2013} 
Titolare di \emph{Assegno di Ricerca} sul tema ``Metodologie ed algoritmi per la robotica mobile di servizio''
presso il Dipartimento di Ingegneria dell'Informazione dell'Universit\`a di Parma.

\ITEMDATE{Gennaio-Febbraio 2009} 
\emph{Prestazione d'opera autonoma occasionale} sul tema ``Interfaccia grafica di 
programmazione per la formazione di strati di prodotto mediante manipolatore'' 
nell'ambito della convenzione tra Universit\`a degli Studi di Parma e \emph{OCME S.r.l.}.

\ITEMDATE{Ottobre 2005 - Giugno 2006}
Borsa di studio sul tema ``Metodi e modelli per il software 
industriale ed in tempo reale'' presso %nell'ambito del laboratorio \emph{LARER} 
il Dipartimento di Ingegneria dell'Informazione dell'Universit\`a di Parma.
%In particolare l'attivit\`a, continuata anche dopo la scadenza della suddetta 
%borsa, ha riguardato il sottoprogetto ``Sviluppo di sistemi robotici ad elevata interazione'',
%OR 7 ``Interfacce evolute per l'interazione con l'ambiente e robot mobili''

\end{itemize}

% ---------------------------------------------------------
% ESTERO
% ---------------------------------------------------------

\section*{Periodi di Ricerca all'Estero}

\begin{itemize}

\ITEMDATE{Luglio-Dicembre 2007} 
Durante il Dottorato di Ricerca \`e stato ospite in qualit\`a di visiting student 
dell'\textit{Institut f\"ur Informatik} della \textit{Albert-Ludwigs Universit\"at} 
di Freiburg (Germania) sotto la supervisione del prof. Wolfram Burgard.

\end{itemize}

% ---------------------------------------------------------
% PROGETTI
% ---------------------------------------------------------

\section*{Partecipazione a Progetti di Ricerca}

\begin{itemize}

\ITEMDATE{Febbraio 2013-Luglio 2016. PRIN MARIS (Marine Autonomous Robotics for InterventionS)} 
Dario Lodi Rizzini partecipa al Progetto di Interesse Narionale MARIS in qualit\`a di membro
dell'unit\`a di ricerca UNIPR dell'Universit\`a degli Studi di Parma. 
L'attivit\`a si \`e concentrata sulla preparazione e svolgimento degli esperimenti
di acquisizione di immagini stereoscopiche e sullo sviluppo di algoritmi di riconoscimento 
oggetti e di stima della posa in ambiente subacqueo.

\ITEMDATE{Giugno 2014-Marzo 2015. FP7 EuRoC (European Robotics Challenges), \\
           Challenge 1 ``Reconfigurable Interactive Manufacturing Cell''} 
Dario Lodi Rizzini \`e stato membro del team Ghepard costituito da partner dell'Universit\`a degli 
Studi di Parma e dell'Universit\`a degli Studi di Genova, che ha partecipato allo Stage 1 della 
Challenge 1 del progetto europeo EuRoC qualificandosi al $2^o$ posto su 10 team qualificati 
(su un totale di circa 30 team iscritti).
I partecipanti allo Stage 1 di EuRoC hanno svolto task di riconoscimento gesti, percezione e presa 
di oggetti in un ambiente industriale simulato. 
In particolare, si \`e occupato della percezione dell'oggetto di interesse e della stima della posa.

\ITEMDATE{Gennaio 2011-Giugno 2012. Progetto Integrapack} 
Ha preso parte al progetto \emph{Integrapak} promosso dalla Regione Emilia-Romagna partecipato 
dell'Universit\`a degli Studi di Parma
e avente come partner industriali OCME S.r.l. e PROMAG S.p.A.
Si \`e occupato della simulazione di un sistema di macchine automatiche per la stima delle prestazioni
e di gestione dei programmi associati a formati differenti. 

\ITEMDATE{Marzo 2009-Dicembre 2010. Laboratorio AERTech}
Ha partecipato al Laboratorio AERTech promosso dalla regione Emilia-Romagna nell'unit\`a di ricerca
dell'Universit\`a degli Studi di Parma.
In particolare, ha dato contributi agli Obiettivi Realizzativi (OR)
OR 3.1 ``Architetture computazionali per la supervisione e il controllo di macchine automatiche'' e
OR 3.3 ``Sistemi robotici per l'ausilio all'uomo''.
In questo ambito si \`e occupato della simulazione e programmazione di macchine automatiche 
(ad esempio, macchina formatore strato pallet) e di metodi di localizzazione e mapping 
di robot mobili industriali impiegati nella logistica. 

\ITEMDATE{Ottobre 2005-Giugno 2007. Laboratorio per l'Automazione della Regione Emilia-Romagna (LARER)}
Ha partecipato al Laboratorio per l'Automazione della Regione Emilia-Romagna (LARER) 
come membro dell'unit\`a di ricerca dell'Universit\`a degli Studi di Parma.
Si \`e occupato del Work Package ``Sviluppo di sistemi robotici ad elevata interazione'',
ed in particolare dell'Obiettivo Realizzativo 7 ``Interfacce evolute per l'interazione con l'ambiente e robot mobili''.
Ha sviluppato algoritmi di localizzazione globale e di mapping adatti all'esecuzione in linea.

\end{itemize}

\section*{Partecipazione a Progetti di Ricerca in convenzione con Aziende Private}

\begin{itemize}

\ITEMDATE{Contratto di Ricerca UNIPR - Elettric80 S.p.A.} 
Titolo: ``Navigazione Ambientale di Robot Mobili con Laser Range Finder'' (Gennaio 2014-Dicembre 2016). 
L'obiettivo \`e lo studio e la realizzazione di metodi per la localizzazione di AGV industriali 
senza impiegare landmark artificiali come invece avviene nei sistemi commerciali.
Dario Lodi Rizzini ha contribuito alla proposta ed allo sviluppo di metodi ed algoritmi 
collaborando con i dottorandi Fabjan Kallasi e Fabio Oleari.

\ITEMDATE{Contratto di Ricerca UNIPR - Elettric80 S.p.A.} 
Titolo: ``Tecniche avanzate di percezione per AGV mediante elaborazione in tempo reale di profili sensoriali generati da laser scanner di sicurezza'' (Gennaio 2015-Settembre 2015). 
L'obiettivo \`e l'analisi per il riconoscimento di oggetti in movimento utilizzando 
i sensori a tecnologia laser al fine di individuare potenziali situazioni di pericolo.
Dario Lodi Rizzini ha contribuito alla proposta ed allo sviluppo di metodi ed algoritmi 
allo stato dell'arte.

\ITEMDATE{Contratto di Ricerca UNIPR - OCME S.r.l.}
La collaborazione ha avuto luogo nel periodo Gennaio 2009-Dicembre 2009. 
Scopo del progetto \`e stato realizzare ed ottimizzare strumenti software innovativi per 
la programmazione, simulazione e supervisione di sistemi di palettizzazione.
Dario Lodi Rizzini ha proposto l'architettura generale del simulatore, la gestione 
ottimizzata di collisioni ed i metodi di generazione della geometria dello strato pallet.

\end{itemize}

% ---------------------------------------------------------
% COLLABORAZIONI RIVISTE E CONFERENZE
% ---------------------------------------------------------

\section*{Partecipazione a Competizioni Internazionali}

Dario Lodi Rizzini ha coordinato la squadra costituita da studenti che ha partecipato a tre edizioni del Sick Robot Day,
competizione di robot mobili aperta ad universit\`a e istituzioni didattiche promossa e sponsorizzata 
da Sick AG, azienda leader mondiale nella produzione di sensori ed in particolare a tecnologia laser. 

\begin{itemize}

\ITEMDATE{Sick Robot Day 2014}
Dario Lodi Rizzini ha coordinato la squadra composta da studenti dell'Universit\`a degli Studi di Parma
che ha partecipato a Walkirch 11 ottobre 2014 al \emph{Sick Robot Day}.
La squadra si \`e classificata al \underline{$1^o$ posto} su 15 team provenienti da Germania, Repubblica Ceca,
Inghilterra ed Italia. 

\ITEMDATE{Sick Robot Day 2012}
Dario Lodi Rizzini ha coordinato la squadra composta da studenti dell'Universit\`a degli Studi di Parma
che ha partecipato a Walkirch 6 ottobre 2012 al \emph{Sick Robot Day}.
La squadra si \`e classificata al \underline{$1^o$ posto} su 15 team provenienti da Germania, Repubblica Ceca,
ed Italia. 

\ITEMDATE{Sick Robot Day 2010}
Dario Lodi Rizzini ha coordinato la squadra composta da studenti dell'Universit\`a degli Studi di Parma
che ha partecipato a Walkirch 2 ottobre 2010 al \emph{Sick Robot Day}.
La squadra si \`e classificata al \underline{$5^o$ posto} su 16 team provenienti da Germania, Repubblica Ceca,
ed Italia. 

\end{itemize}

% ---------------------------------------------------------
% COLLABORAZIONI RIVISTE E CONFERENZE
% ---------------------------------------------------------

\section*{Partecipazione a comitati di Riviste e Conferenze e attivit\`a ulteriore di revisione scientifica}

\subsubsection*{Membro di Program Committee e Session Chair}

\begin{itemize}
\ITEMDATE{Membro del Program Committee di ECMR 2011}
Dario Lodi Rizzini \`e stato invitato come membro del Program Committee della European Conference on Mobile Robotics (ECMR) 2012
tenutasi a Orebro (Svezia). 
\end{itemize}

Dario Lodi Rizzini ha svolto il compito di Session Chair alle seguenti conferenze.
\begin{itemize}
\item IEEE/MTS OCEANS, Genova (IT), 2015. 
\item IEEE/RSJ Int.~Conf.~on Intelligent Robots and Systems (IROS), St. Louis (USA), 2010.
\item Intl.~Conf.~on Informatics in Control, Automation and Robotics (ICINCO), Anger (FR), 2007.
\end{itemize}


\subsubsection*{Revisore}

Dario Lodi Rizzini \`e stato revisore per le seguenti riviste internazionali:
\begin{itemize}
\item IEEE~Transaction~on Robotics (T-RO).
\item IEEE~Transaction~on Automation Science and Engineering (T-ASE).
\item Springer Robotics and Autonomous Systems (RAS).
\item Springer Mechatronics.
\item Springer International Journal on~Control, Automation and Systems (IJCAS).
\end{itemize}

\noindent \`E stato revisore per le seguenti conferenze internazionali:
\begin{itemize}
\item IEEE Int.~Conf.~on Robotics \& Automation (ICRA).
\item IEEE/RSJ Int.~Conf.~on Intelligent Robots and Systems (IROS).
\item Int.~Conf.~on Advanced Robotics (ICAR).
\item European Conference on Mobile Robots (ECMR).
\item AAAI Conference on Artificial Intelligence. 
\item Spacial Cognition 2012.
\item IEEE Intelligent Transportation Systems Society Conference Management System (ITSC).
\item IEEE Control Systems Society Conference Management System (CDC).
\item PID'12 Conference.
\end{itemize}

\subsection*{Presentazioni Orali}

Dario Lodi Rizzini ha presentato oralmente i propri lavori alle seguenti conferenze o workshop.
\begin{itemize}
\item 2015: OCEANS, Genova (IT).
\item 2014: World Congr. of IFAC, Capetown (ZA).
\item 2014: IAS e workshop AMRA, Padova (IT).
\item 2013: ECMR, Barcelona (ES). 
\item 2012: ICRA Workshop, St. Paul (USA). 
\item 2012: ICINCO, Roma (IT). 
\item 2011: ICRA, Shangai (CN). 
\item 2010: IROS and Graphbot Workshop, Taipei (TW).
\item 2009: ECMR, Dubrovnik (HR).
\item 2009: IROS, St. Louis (USA).
\item 2009: ICAR, Munich (DE).
\item 2009: ICRA Workshop, Kobe (JP).
\item 2008: CIRAS, Linz (JP).
\item 2007: ECMR, Freiburg (DE).
\item 2007: ICINCO, Angers (FR).
\end{itemize}

% ---------------------------------------------------------
% ATTIVITA DIDATTICA
% ---------------------------------------------------------

%\section*{Attivit\`a Organizzativa e di Coordinamento per l'Ateneo}

%\begin{itemize}
%\ITEMDATE{}
%Dario Lodi Rizzini \`e stato ha collaborato alla sorvegl
%\end{itemize}

\section{Attivit\`a Didattica}

Dario Lodi Rizzini ha svolto attivit\`a didattica nell'ambito di insegnamenti del 
SSD ING/INF-05 ``Sistemi di elaborazione delle informazioni'' presso la Facolt\`a di Ingegneria 
sino all'A.A. 2011-2012 e successivamente presso il Dipartimento di Ingegneria dell'Informazione 
dell'Universit\`a degli Studi di Parma. 

\subsection*{Contratti di Insegnamento}

\begin{itemize}
\ITEMDATE{A.A. 2012-2015}
Dario Lodi Rizzini \`e stato \underline{docente a contratto} dell'insegnamento di ``Robotica'' (SSD ING-INF/05, 6 CFU, 44 ore)
per il corso di Laurea Magistrale in Ingegneria Informatica del Dipartimento di Ingegneria dell'Informazione 
dell'Universit\`a degli Studi di Parma.

\ITEMDATE{A.A. 2011-2012}
Dario Lodi Rizzini \`e stato \underline{docente a contratto} dell'insegnamento di ``Robotica'' (SSD ING-INF/05, 6 CFU, 44 ore)
per il corso di Laurea Magistrale in Ingegneria Informatica della Facolt\`a di Ingegneria 
dell'Universit\`a degli Studi di Parma.

\ITEMDATE{A.A. 2010-2011}
Dario Lodi Rizzini \`e stato \underline{docente a contratto} del Modulo 2 dell'insegnamento di ``Robotica'' 
(SSD ING-INF/05, 3 CFU, 22 ore) per il corso di Laurea Magistrale in Ingegneria Informatica 
della Facolt\`a di Ingegneria dell'Universit\`a degli Studi di Parma.
Il Modulo 1 dell'insegnamento di ``Robotica'' (3 CFU) \`e stato affidato all'ing. Jacopo Aleotti.

\ITEMDATE{A.A. 2009-2010}
Dario Lodi Rizzini \`e stato \underline{docente a contratto} del Modulo 2 dell'insegnamento di ``Robotica'' 
(SSD ING-INF/05, 2 CFU, 16 ore) per il corso di Laurea Magistrale in Ingegneria Informatica della Facolt\`a di Ingegneria.
Il Modulo 1 dell'insegnamento di ``Robotica''  (3 CFU) \`e stato affidato all'ing. Jacopo Aleotti.

\ITEMDATE{A.A. 2009-2010}
Contratto per attivit\`a didattica integrativo per il modulo di
``Sistemi Operativi A'' (SSD ING-INF/05) per il corso di Laurea in Ingegneria Informatica con didattica a 
distanza in Ingegneria dell'Universit\`a degli Studi di Parma.
\end{itemize}

\subsection*{Attivit\`a di Sostegno alla Didattica}

\begin{itemize}

\ITEMDATE{A.A. 2009-2015} 
Attivit\`a didattica di sostegno all'insegnamento di ``Sistemi Operativi ed in Tempo Reale'' 
(SSD ING-INF/05, 9 CFU, titolare: prof. Stefano Caselli)
per i corsi di Laurea Magistrale in Ingegneria Informatica
della Facolt\`a di Ingegneria dell'Universit\`a degli Studi di Parma.

\ITEMDATE{A.A. 2006-2009} 
Attivit\`a didattica di sostegno all'insegnamento di ``Sistemi Operativi B'' (SSD ING-INF/05, 5 CFU, titolare: prof. Stefano Caselli)
per i corsi di Laurea Specialistica in Ingegneria Informatica, in Ingegneria Elettronica ed in Ingegneria delle Telecomunicazioni
della Facolt\`a di Ingegneria dell'Universit\`a degli Studi di Parma.

\ITEMDATE{A.A. 2006-2009} 
Attivit\`a didattica di sostegno all'insegnamento di ``Robotica'' (SSD ING-INF/05, 5 CFU, titolare: prof. Stefano Caselli)
per il corso di Laurea Specialistica in Ingegneria Informatica 
della Facolt\`a di Ingegneria dell'Universit\`a degli Studi di Parma.

\ITEMDATE{A.A. 2006-2007} 
Contratto per attivit\`a di tutorato per il corso di ``Sistemi Operativi B'' (SSD ING-INF/05, 5 CFU, titolare: prof. Stefano Caselli)
per i corsi di Laurea Specialistica in Ingegneria Elettronica, in Ingegneria Informatica ed in Ingegneria delle Telecomunicazioni 
della Facolt\`a di Ingegneria dell'Universit\`a degli Studi di Parma.

\ITEMDATE{Marzo-Settembre 2005}
Contratto per attivit\`a di tutorato per il corso di ``Controlli Automatici A'' (SSD ING-INF/04, 5 CFU, titolare: prof. Aurelio Piazzi) 
per i corsi di Laurea in Ingegneria Elettronica, in Ingegneria Informatica ed in Ingegneria delle Telecomunicazioni 
della Facolt\`a di Ingegneria dell'Universit\`a degli Studi di Parma. 
\end{itemize}

\subsection*{Correlatore di Tesi di Laurea}

Dario Lodi Rizzini \`e stato correlatore di pi\`u di 20 tesi di Laurea, Laurea Magistrale e Specialistica.

% ---------------------------------------------------------
% PUBBLICAZIONI
% ---------------------------------------------------------

\section{Elenco delle Pubblicazioni}

\noindent\textbf{Riviste Internazionali:}

\begin{itemize}
\BIBCITEM{lodirizzini2015ijars}
\BIBCITEM{aleotti2014jirs}
\BIBCITEM{cigolini2014jamris}
\BIBCITEM{lodirizzini2009ras}
\end{itemize}

% Conferenze
\noindent\textbf{Atti di Conferenze Internazionali con revisione su articolo completo:}

\begin{itemize}
%
\BIBCITEM{kallasi2015oceans}
\BIBCITEM{oleari2015oceans}
%
\BIBCITEM{aleotti2014iros}
\BIBCITEM{oleari2014ifac}
\BIBCITEM{lodirizzini2014ias}
\BIBCITEM{kallasi2014iarp}
%
\BIBCITEM{lodirizzini2013ecmr}
\BIBCITEM{valeriani2013acra}
\BIBCITEM{oleari2013iccp}
%
\BIBCITEM{aleotti2012icra}
\BIBCITEM{lodirizzini2012icinco}
\BIBCITEM{calo2012icinco}
%
\BIBCITEM{lodirizzini2011icra}
%
\BIBCITEM{lodirizzini2010iros}
\BIBCITEM{argenti2010isr}
%
\BIBCITEM{lodirizzini2009iros}
\BIBCITEM{lodirizzini2009ecmr}
\BIBCITEM{lodirizzini2009icar}
%
\BIBCITEM{grisetti2008icra}
\BIBCITEM{lodirizzini2008ciras}
%
\BIBCITEM{lodirizzini2007ecmr}
\BIBCITEM{lodirizzini2007icinco}
\end{itemize}

% Workshop
\noindent\textbf{Atti di Workshop con revisione basata su abstract:}

\begin{itemize}
\BIBCITEM{kallasi2014amra}
%
\BIBCITEM{lodirizzini2012icraworkshop}
%
\BIBCITEM{lodirizzini2010graphbot}
%
\BIBCITEM{lodirizzini2009icraworkshop}
%
\BIBCITEM{cerri2007ccmvs}
\end{itemize}


% Libri
\noindent\textbf{Capitoli di Libri:}

\begin{itemize}
\BIBCITEM{lodirizzini2008improved}
\end{itemize}

% Libri
\noindent\textbf{Tesi di Dottorato di Ricerca:}

\begin{itemize}
\BIBCITEM{lodirizzini2009thesis}
\end{itemize}

~\\
~\\
\vfill
\begin{minipage}[t]{7.0cm}
  \raggedright
  \parbox{7.0cm}{\centering
    Luogo e Data, \makebox[3.0cm]{\hrulefill} %$\frac{\hspace{2.5cm}}{\hspace{2.5cm}}$
  }\\
\end{minipage}%
\hfill
\begin{minipage}[t]{7.0cm}
  \parbox{7.0cm}{\centering
    Firma \makebox[4.0cm]{\hrulefill}   %    $\frac{\hspace{6cm}}{\hspace{6cm}}$
  }
\end{minipage}


%\nobibliography*
%\bibliographystyle{unsrt}
%\bibliographystyle{abbrv}
%\bibliographystyle{rimlab}
\nobibliography{dlr_pubblications}

\end{document}

