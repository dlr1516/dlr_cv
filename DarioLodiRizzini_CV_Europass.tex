% !TEX encoding = UTF-8
% !TEX program = pdflatex
% !TEX spellcheck = en_GB

\documentclass[english,a4paper]{europasscv}
\usepackage[english]{babel}
\usepackage{notoccite}
\usepackage{bibentry}
\makeatletter\let\saved@bibitem\@bibitem\makeatother
%\usepackage{hyperref}
%\makeatletter\let\@bibitem\saved@bibitem\makeatother
%\usepackage[backend=bibtex, style=numeric, sorting=ynt, defernumbers, maxnames=99]{biblatex}
%\usepackage{europasscv-bibliography}


%\usepackage[backend=biber,autolang=hyphen,sorting=none,style=numeric,maxbibnames=99,doi=false,isbn=false,maxcitenames=2]{biblatex}
%\usepackage{csquotes}
%\usepackage{europasscv-bibliography}
%\bibliography{dlr_pubblications}


\newcommand{\ITEMDATE}[1]{\item \textit{#1:}\\}
\newcommand{\BIBCITE}[1]{\cite{#1} \bibentry{#1}}
%\newcommand{\BIBCITEM}[1]{\item[\cite{#1}] \bibentry{#1}}
%\newcommand{\BIBCITEM}[1]{\ecvitem{ \cite{#1} }{ \bibentry{#1} }}
\newcommand{\BIBCITEM}[1]{\ecvitem{ \cite{#1} }{ \begin{NoHyper}\bibentry{#1}\end{NoHyper} }}  


\ecvname{Dario Lodi Rizzini}
\ecvaddress{Work Address: 
  viale Parco Area delle Scienze, 181A I-43124 Parma Italy}
\ecvmobile{+39 333 6161152}
%\ecvtelephone{+353 127 6689}
\ecvworkphone{+39 0521 906147}
\ecvemail{dario.lodirizzini@unipr.it, dario.lodirizzini@pec.it (PEC)}
\ecvhomepage{http://rimlab.ce.unipr.it/LodiRizzini.php}
\ecvgithubpage{https://github.com/dlr1516}
% \ecvgitlabpage{www.gitlab.com/smith}
% \ecvlinkedinpage{www.linkedin.com/in/katie-smith}
% \ecvorcid[label, link]{0000-0000-0000-0000}
%\ecvim{AOL Messenger}{katie.smith}
%\ecvim{Google Talk}{ksmith}

% \ecvgender{Female}
\ecvdateofbirth{23 May 1981}
\ecvnationality{Italy}

% \ecvpicture[width=3.8cm]{picture.jpg}

% \date{}

\begin{document}
  \begin{europasscv}

%\bibliographystyle{unsrt_dlr}
%\nocite{lodirizzini2019ral,galasso2019rcim,
%lodirizzini2018pr,simetti2018joe,
%kallasi2017ras,lodirizzini2017caee, 
%casalino2016mtsj, kallasi2016ral, 
%lodirizzini2015ijars, 
%aleotti2014jirs, cigolini2014jamris, 
%lodirizzini2009ras, 
%lodirizzini2018irosws,
%lodirizzini2017iros, 
%kallasi2016iros, oleari2016iecon,
%kallasi2015oceans, oleari2015oceans,
%aleotti2014iros, oleari2014ifac, lodirizzini2014ias, kallasi2014iarp,
%lodirizzini2013ecmr, valeriani2013acra, oleari2013iccp, 
%aleotti2012icra, lodirizzini2012icinco, calo2012icinco, 
%lodirizzini2011icra, 
%lodirizzini2010iros, argenti2010isr, 
%lodirizzini2009iros, lodirizzini2009ecmr, lodirizzini2009icar, 
%grisetti2008icra, lodirizzini2008ciras, 
%lodirizzini2007ecmr, lodirizzini2007icinco, 
%%
%kallasi2014amra, lodirizzini2012icraworkshop, lodirizzini2010graphbot, lodirizzini2009icraworkshop, cerri2007ccmvs,
%%
%lodirizzini2008improved,
%lodirizzini2009thesis}

  \bibliographystyle{unsrt_dlr}
  \nobibliography{dlr_pubblications}


\nocite{
monica2024tvcg,
penzotti2023compag,fontana2023sensors,
lodirizzini2022ral,
aleotti2021ap,%consolini2020pr,
lodirizzini2019ral,galasso2019rcim,
lodirizzini2018pr,simetti2018joe,
kallasi2017ras,lodirizzini2017caee, 
casalino2016mtsj, kallasi2016ral, 
lodirizzini2015ijars, 
aleotti2014jirs, cigolini2014jamris, 
lodirizzini2009ras,
monica2024etfa,
penzotti2023ecpa,
khan2022irc,lodirizzini2022raad,
fontana2021iccp,khan2021iccp,fontana2021ecmr,
monica2020irc,chiaravalli2020etfa,amoretti2020smartsys,
lodirizzini2018irosws,
lodirizzini2017iros, 
kallasi2016iros, oleari2016iecon,
kallasi2015oceans, oleari2015oceans,
aleotti2014iros, oleari2014ifac, lodirizzini2014ias, kallasi2014iarp,
lodirizzini2013ecmr, valeriani2013acra, oleari2013iccp, 
aleotti2012icra, lodirizzini2012icinco, calo2012icinco, 
lodirizzini2011icra, 
lodirizzini2010iros, argenti2010isr, 
lodirizzini2009iros, lodirizzini2009ecmr, lodirizzini2009icar, 
grisetti2008icra, lodirizzini2008ciras, 
lodirizzini2007ecmr, lodirizzini2007icinco, 
%
fontana2023icraws,
lodirizzini2021simai,
kallasi2014amra, lodirizzini2012icraworkshop, lodirizzini2010graphbot, lodirizzini2009icraworkshop, cerri2007ccmvs,
%
lodirizzini2008improved,
lodirizzini2009thesis}

  \ecvpersonalinfo


%  \ecvbigitem{Job applied for}{Principal Investigator}

% ---------------------------------------------------------
% WORK EXPERIENCE
% ---------------------------------------------------------


  \ecvsection{Work experience}

  \ecvtitle{February 2024 -- Present}{Associate Professor}
  \ecvitem{}{ (Professore di Fascia II) }
  \ecvitem{}{Dipartimento di Ingegneria e Architettura, Universit\`a degli Studi di Parma}

  \ecvtitle{February 2021 -- January 2024}{Assistant Professor}
  \ecvitem{}{ (Ricercatore a Tempo Determinato - tipo B) }
  \ecvitem{}{Dipartimento di Ingegneria e Architettura, Universit\`a degli Studi di Parma}

  \ecvtitle{December 2015 -- 2020}{Assistant Professor}
  \ecvitem{}{ (Ricercatore a Tempo Determinato - tipo A) }
  \ecvitem{}{Dipartimento di Ingegneria e Architettura, Universit\`a degli Studi di Parma}

  \ecvtitle{June 2013 -- December 2015}{Post-Doctoral Researcher}
  \ecvitem{}{ (Assegno di Ricerca) }
  \ecvitem{}{Dipartimento di Ingegneria dell'Informazione, Universit\`a degli Studi di Parma}
  \ecvitem{}{Topic: ``Probabilistic Methods for object detection in robot manipulation and navigation tasks''}

  \ecvtitle{March 2009 -- April 2013}{Post-Doctoral Researcher}
  \ecvitem{}{(Assegno di Ricerca)}
  \ecvitem{}{Dipartimento di Ingegneria dell'Informazione, Universit\`a degli Studi di Parma}
  \ecvitem{}{Topic: ``Methods and algorithms for service Robotics''}

  \ecvtitle{January-February 2009}{Independent consultant}
  \ecvitem{}{Research contract between the Universit\`a degli Studi di Parma and \emph{OCME S.r.l.}}
  \ecvitem{}{Topic ``Graphical Interface for programming of industrial palletizing machine based on robot manipulation}

  \ecvtitle{October 2005 -- June 2006}{Scholarship}
  \ecvitem{}{Scholarship for project LARER - Regione Emilia-Romagna}
  \ecvitem{}{Topic: ``Methods and models for industrial real-time software''}

%  
%  \ecvtitle{August 2002 -- Present}{Independent consultant}
%  \ecvitem{}{National Youth Council of Ireland\newline 3 Montague Street, Dublin 2, D02 V327, Ireland}
%  \ecvitem{}{Evaluation of European Commission youth training support measures for youth national agencies and young people}
%   
%  \ecvtitle{March 2002 -- July 2002}{Internship}
%  \ecvitem{}{European Commission, Youth Unit, DG Education and Culture \newline 200, Rue de la Loi, 1049 Brussels (Belgium)}
%  \ecvitem{}{
%  \begin{ecvitemize}
%      \item evaluating youth training programmes and the partnership between the Council of Europe and European Commission
%      \item organizing and running a 2 day workshop on non-formal education for Action 5 large scale projects focusing on quality, assessment and recognition
%      \item contributing to the steering sroup on training and developing action plans on training for the next 3 years. Working on the Users Guide for training and the support measures
%  \end{ecvitemize}
%  }
%  \ecvitem{}{\ecvhighlight{Business or sector}\quad European institution}
%  
%  \ecvtitle{Oct 2001 -- Feb 2002}{Researcher / Independent Consultant}
%  \ecvitem{}{Council of Europe, Budapest (Hungary)}
%  \ecvitem{}{Working in a research team carrying out in-depth qualitative evaluation of the 2 year Advanced Training of Trainers in Europe using participant observations, in-depth interviews and focus groups. Work carried out in training courses in Strasbourg, Slovenia and Budapest.}
  
  
% ---------------------------------------------------------
% EDUCATION
% ---------------------------------------------------------

  \ecvsection{Education and training}
  
  \ecvtitle{January 2006--March 2009}{Doctor of Philosophy (PhD) in Information Technology}
  \ecvitem{}{(Dottorato di Ricerca in Tecnologie dell'Informazione, XXI ciclo)}
  \ecvitem{}{Universit\`a degli Studi di Parma, Italy}
  \ecvitem{}{Dissertation title: ``Computation and Time Constraints in Localization and Mapping Problems''}
  \ecvitem{}{Thesis defense on March 3rd 2009}
  %\ecvitem{}{Trinity College Dublin, The University of Dublin, Ireland}

  \ecvtitle{July-December 2007}{Visiting PhD Student}
  \ecvitem{}{ \emph{Institut f\"ur Informatik}, \emph{Albert-Ludwigs Universit\"at}, Freiburg, Germany }
  \ecvitem{}{ Supervisor: prof. Wolfram Burgard }
  
  \ecvtitle{October 2003--December 2005}{Master Degree in in Computer Engineering}
  \ecvitem{}{(Laurea Specialistica in Ingegneria Informatica)}
  \ecvitem{}{Universit\`a degli Studi di Parma, Italy}
  \ecvitem{}{Thesis Title: ``Progettazione di una Libreria per la Localizzazione e Fusione Sensoriale basata su Filtri Particellari''}
  \ecvitem{}{Final degree mark: 110/100 cum laude}
  \ecvitem{}{Graduation date: December 15th 2005}

  \ecvtitle{October 2000--September 2003}{Bachelor Degree in in Computer Engineering}
  \ecvitem{}{(Laurea in Ingegneria Informatica)}
  \ecvitem{}{Universit\`a degli Studi di Parma, Italy}
  \ecvitem{}{Thesis Title: ``La Localizzazione in interni tramite una Rete Wireless Ethernet''}
  \ecvitem{}{Final degree mark: 110/100 cum laude}
  \ecvitem{}{Graduation date: September 25th 2005}
  

% ---------------------------------------------------------
% EDUCATION
% ---------------------------------------------------------

  \ecvsection{Academic Qualifications}

  \ecvtitle{September 2019--September 2028}{Abilitazione Scientifica Nazionale (ASN) for Professore II Fascia}
  \ecvitem{}{ Bando D.D. 1532/2016, Sector 09/H1, SSD ING-INF/05 }


% ---------------------------------------------------------
% RESEARCH
% ---------------------------------------------------------

  \ecvsection{Research Interests}

  \ecvtitle{}{Robot Localization and Mapping} 
  \ecvitem{}{I have worked on robot localization and mapping algorithms since my PhD program. 
    In particular, I have studied Graphical SLAM formulation for map estimation by proposing an incremental stochastic-gradient descent solver~\cite{grisetti2008icra, lodirizzini2009ras} with the group from the University of Freiburg, a decomposition of the graph-map suitable for parallel~\cite{lodirizzini2009iros} and multi-robot estimation~\cite{lodirizzini2010iros, lodirizzini2010graphbot}, and an algorithm for correcting data association errors~\cite{lodirizzini2011icra}.
    The papers~\cite{lodirizzini2009ecmr,lodirizzini2009icar} are among the first analyses about the structure of pose graph formulation and its possible closed-form solution. 
  }
  \ecvitem{}{I proposed an extension of Real-Time Particle Filter (RTPF)~\cite{lodirizzini2007icinco, lodirizzini2007ecmr, lodirizzini2008improved}, a variant of the Montecarlo robot localization algorithms suitable for real-time execution.}
  \ecvitem{}{Several of my works address the issues related to the integration of sensor data into a consistent representation. 
    The paper~\cite{kallasi2016ral} proposes FALKO, one of the first effective keypoint features from laser scans to build a landmark map. 
    The signatures GLAROT (Geometric Landmark Relation Orientation-invariant)~\cite{kallasi2016iros, lodirizzini2017iros} and GRD (Geometric Relation Distribution)~\cite{lodirizzini2019ral} effectively address the loop closure problem for landmark maps. 
    I proposed Angular Randon Spectrum (ARS)~\cite{lodirizzini2018pr, lodirizzini2022ral, fontana2023sensors} to solve map registration problem by decoupling the estimation of rotation and translation. 
    The ARS provides a formally sound description of collinearity.
    %GLORES algorithm~\cite{consolini2020pr} presents one of the first globally optimal registration algorithms for alignment of planar point maps. 
  }

  \ecvtitle{}{Advanced Perception}  
  \ecvitem{}{I have extensively worked on 3D perception and point cloud processing using computer vision~\cite{lodirizzini2017caee, lodirizzini2015ijars, kallasi2015oceans,
oleari2014ifac, oleari2013iccp}, LIDARs~\cite{aleotti2014jirs, aleotti2012icra} and depth cameras~\cite{monica2020irc, chiaravalli2020etfa, aleotti2014iros, lodirizzini2014ias}. 
    Depending on the specific application, perception has been applied to object recognition, pose estimation and manipulation. 
    The problem addressed in~\cite{aleotti2012icra, aleotti2014jirs} is the complete observation, reconstruction and segmentation of objects, which is achieved through next-best view planning (NBV) and integration of multiple sensor data~\cite{aleotti2014iros}. 
  }
  \ecvitem{}{The work in~\cite{lodirizzini2014ias} presents the segmentation and classification of voxel maps representing the environment whereas in~\cite{oleari2013iccp} we present object recognition and registration using low-cost stereo vision and keypoint features.
  }
  \ecvitem{}{I have contributed to the vision system of MARIS project, whose goal was the cooperative manipulation of AUVs (Autonomous Underwater Vehicle). 
    I and my coworkers proposed several approaches to segmentation, object recognition~\cite{lodirizzini2015ijars, kallasi2015oceans} and geometric pose estimation~\cite{lodirizzini2017caee}. 
   The developed solutions are carefully crafted to the difficult sensing conditions and to the accurate manipulation requirements. 
  }

  \ecvtitle{}{Field Robotics and Industrial Applications}  
  \ecvitem{}{A significant portion of my research activity has been oriented toward robotic applications.
    The perception system implemented for the previously discussed MARIS project was combined with extensive experiments to collect datasets~\cite{oleari2015oceans, oleari2014ifac, kallasi2014amra}, to build water-tight and electronically safe sensor systems~\cite{oleari2016iecon} and to integrate the sensors with the UAV system~\cite{simetti2018joe}.
  }
  \ecvitem{}{
    I have worked on several projects related to industrial automation and robotics, often for food industry applications~\cite{aleotti2021ap, monica2020irc, chiaravalli2020etfa, fontana2021ecmr, fontana2021iccp}. 
    I have contributed to the development of a software tool for the simulation, programming and product format generation of an industrial palletizing machine~\cite{argenti2010isr, calo2012icinco}. 
    Robot planning methodologies have been applied to agriculture to improve irrigation efficiency~\cite{penzotti2023compag, penzotti2023ecpa, amoretti2020smartsys} or fruit picking and manipulation~\cite{monica2024etfa}.    
  }
  \ecvitem{}{
    Another recurrent field is the navigation of industrial AGVs (Automated Guided Vehicle) adopted in warehouse logistics~\cite{lodirizzini2022raad}. 
    The paper in~\cite{lodirizzini2012icinco} presents an algorithm for adapting the AGV path to the presence of semi-static objects. 
    Another relevant contribution is the automatic calibration of intrinsic (kinematic) and extrinsic parameters of AGVs, which is required to precisely reach operation points (e.g. deposit and loading of pallets). 
    I proposed complete calibration methods for different kinematics including tricycle~\cite{kallasi2017ras} and four-wheel AGVs~\cite{galasso2019rcim}. 
    I also developed advanced and robust localization and navigation systems for AGVs~\cite{lodirizzini2018irosws}, which do not rely on artificial landmarks. 
    Most of these contributions are now standard part of the industrial plants built by the industrial partners. 
  }
  \ecvitem{}{
    Other contributions are related to academic experiences. 
    Significant examples are the robotic systems developed for robot competitions like Sick Robot Day organized by sensor manufacturer~\cite{cigolini2014jamris, valeriani2013acra}. 
  }


% ---------------------------------------------------------
% RESEARCH PROJECTS
% ---------------------------------------------------------

  \ecvsection{Research Projects}

\ecvtitle{2023 -- Present}{POR-FESR 2021-2027 AGRARIAN}
  \ecvitem{}{Project funded by Regione Emilia Romagna}
  \ecvitem{}{Role: Principal Investigator for UNIPR-CIDEA}

  \ecvtitle{2023 -- Present}{POR-FESR 2021-2027 SIMOD}
  \ecvitem{}{Project funded by Regione Emilia Romagna}
  \ecvitem{}{Role: Developer}

  \ecvtitle{2019 -- 2022}{POR FSE 2014/2020 MAN3DP (Mapping And Navigation based on 3D Perception)}
  \ecvitem{}{Project associated to the PhD scholarship (XXXV ciclo) of PhD student Asad Ullah Khan as part of the program 
    ``Alte competenze per la ricerca e il trasferimento tecnologico'' funded by Regione Emilia Romagna}
  \ecvitem{}{Role: Principal Investigator}
  %\ecvitem{}{Maximum budget: Euro 86743.44}

  \ecvtitle{2019 -- 2022}{POR-FESR 2014/2020 POSITIVE (Protocolli Operativi Scalabili per l'Agricoltura di Precisione)}
  \ecvitem{}{Project funded by Regione Emilia Romagna}
  \ecvitem{}{Role: Coordinator of Task 3 (Work package 4): ``Development of operative scalable protocols for intelligent planning of irrigation machines'' }

  \ecvtitle{March 2019 -- 2022}{POR-FESR 2014/2020 COORSA (COllaborazione tra Operatori e Robot manipolatori mobili Sicuri per la fAbbrica del futuro)}
  \ecvitem{}{Project funded by Regione Emilia Romagna}
  \ecvitem{}{Role: contribution on the development of localization and navigation of mobile robots for robot manipulation in industrial environments }

  \ecvtitle{March 2017 -- April 2018}{FIL 2016 R3D-MAN (Robust 3D Mapping and Navigation)}
  \ecvitem{}{Internal Project of the Universit\`a degli Studi di Parma}
  \ecvitem{}{Role: Principal Investigator}
  \ecvitem{}{Goal: development of algorithms for 3D LIDAR-based perception for mapping and navigation of mobile robots}
  %\ecvitem{}{Budget: Euro 4140.00}

  \ecvtitle{June 2016 -- June 2018}{POR-FESR 2014-2020 Aladin (Agroalimentare Idrointelligente)}
  \ecvitem{}{Project funded by Regione Emilia Romagna}
  \ecvitem{}{Role: contribution on the definitions of maps for irrigation from georeferenced vegetation indices and data}

  \ecvtitle{February 2013 -- July 2016}{PRIN MARIS (Marine Autonomous Robotics for InterventionS)}
  \ecvitem{}{Project funded by Ministero dell'Istruzione, dell'Universit\`a e della Ricerca Scientifica (MIUR)}
  \ecvitem{}{Role: Leader of work package WP2 ``Object detection and grasp planning'' with underwater robots (UAV)}

  \ecvtitle{June 2014 -- March 2015}{FP7 EuRoC (European Robotics Challenges)}
  \ecvitem{}{Challenge 1 ``Reconfigurable Interactive Manufacturing Cell''}
  \ecvitem{}{Role: Member of Ghepard team}
  \ecvitem{}{
  \begin{ecvitemize}
      \item qualified in 2nd place over 10 teams
      \item partnership between the Universities of Genova and Parma
      \item contributing to the gesture recognition and object detection
  \end{ecvitemize}
  }

  \ecvtitle{January 2011 -- June 2012}{Integrapack}
  \ecvitem{}{Funded by Regione Emilia-Romagna - Partnership between OCME S.r.l. and PROMAG S.p.A }
  \ecvitem{}{Role: Consultant on simulation of palletizing machines with different product formats}

  \ecvtitle{March 2009 -- December 2010}{AERTech Laboratory}
  \ecvitem{}{Project funded by Regione Emilia-Romagna}
  \ecvitem{}{Role: contribution to OR 3.1 ``Architectures for supervision and control of automation machines'' and 
    OR 3.3 ``Robotic systems supporting human operators''
  }

  \ecvtitle{October 2005 -- June 2007}{Laboratorio per l'Automazione della Regione Emilia-Romagna (LARER)}
  \ecvitem{}{Project funded by Regione Emilia-Romagna}
  \ecvitem{}{Role: contribution to WP 7 ``Development of robotic systems with high interraction'' by developing global localization algorithms
  }

% ---------------------------------------------------------
% RESEARCH CONTRACTS
% ---------------------------------------------------------

  \ecvsection{Research Contracts}
  
  \ecvtitle{November 2023 -- Present}{Research contract between CIDEA -- E80 Group S.p.A.}
%  \ecvitem{}{Topic: ``Analysis of methods for localization and mapping based on graphical SLAM and degenerate space detection in point cloud registration'' 
%    (``Analisi dei Metodi di Localizzazione e Mapping basati su approccio Graphical SLAM e
%Riconoscimento di Sotto-spazi Degeneri nei problemi di Registrazione di Point Cloud in Ambienti
%Simmetrici''). 
%  }
  \ecvitem{}{Role: Project Coordinator.}

  \ecvtitle{July 2021 -- December 2023}{Research contract between DIA/UNIPR -- OCME S.r.l.}
%  \ecvitem{}{Topic: ``3D perception methods with depth cameras for detection and manipulation of parcel boxes'' 
%    (``Metodi di percezione 3D con sensori di visione e profondità per il riconoscimento e manipolazione di prodotti''). 
%    The goal is the development of parcel box detection in pallet layers and manipulation with collaborative robot manipulators. 
%  }
%  \ecvitem{}{Contract actitvities are funded by MISE\_FCS\_DM 5 marzo 2018, Proposal n. 169 known as MAF ``Macchine Autonome \&
%   Flessibili''.}
  \ecvitem{}{Role: Project Coordinator}
%  \ecvitem{}{Budget: Euro 46500.00}
  
  \ecvtitle{January 2018 -- December 2020}{Research contract between CIDEA/UNIPR -- Elettric80 S.p.A. (3D-PAL)}
%  \ecvitem{}{Topic: ``Analysis and development of 3D perception, localization, mapping and naivgation in industrial environment'' 
%    (``Studio e sviluppo di metodi di percezione, localizzazione, navigazione e mapping 3D in ambienti industriali'') 3DPAL. 
%    The goal is the development of localization algorithms for industrial AGVs and the analysis of the impact of 3D perception and mapping algorithms to AGV navigation. 
%  }
%  \ecvitem{}{Contract actitvities are part of project SIMON CUP E18I17000110009 funded with the program POR-FESR 2014-2020 of Regione Emilia-Romagna}
  \ecvitem{}{Role: Project Coordinator}
%  \ecvitem{}{Budget: Euro 95000.00}

  \ecvtitle{September 2016 -- June 2018}{Research contract between UNIPR -- Elettric80 S.p.A.}
%  \ecvitem{}{Topic: ``Dynamic adaptative algorithms of AGV trajectories based on sensor data in work environments'' 
%    (``Adattamento dinamico di traiettorie di AGV in funzione di rilevazioni sensoriali dell'ambiente di lavoro''). 
%    The output of the contract includes the development of calibration algorithms for four-wheel AGVs and localization algorithms without artificial landmarks. 
%  }
  \ecvitem{}{Role: Advisor of ing. Francesco Galasso in Dottorato Industriale in Alta Formazione XXXII ciclo (Industrial Internship PhD program).}

  \ecvtitle{January 2014 -- December 2016}{Research contract between UNIPR -- Elettric80 S.p.A.}
%  \ecvitem{}{Topic: ``Navigation of mobile robots based on laser range finder'' 
%    (``Navigazione Ambientale di Robot Mobili con Laser Range Finder''). 
%    The output of the contract includes the development of calibration algorithms for tricyle AGVs and accurate positioning of industrial AGVs in block-storages of industrial warehouses. 
%  }
  \ecvitem{}{Role: Co-advisor of PhD students Fabjan Kallasi and Fabio Oleari.}

  \ecvtitle{January 2015 -- September 2015}{Research contract between UNIPR -- Elettric80 S.p.A.}
%  \ecvitem{}{Topic: ``Advanced perception methods for dynamic obstacle detection using safety laser scanners'' 
%    (``Tecniche avanzate di percezione per AGV mediante elaborazione in tempo reale di profili sensoriali generati da laser scanner di sicurezza''). 
%    The output of the contract includes an algorithm for detection of pedestrian and dynamic objects in laser scans to improve safety of AGV navigation. 
%  }
  \ecvitem{}{Role: consultant and developer}

  \ecvtitle{January 2009 -- December 2009}{Research contract between UNIPR -- OCME S.r.l.}
%  \ecvitem{}{The output of the contract includes the development of a software for simulation, programming and computation of layer arrangements for a palletizing machine. 
%  }
  \ecvitem{}{Role: consultant and developer}

% ---------------------------------------------------------
% SCIENTIFIC COMMUNITY
% ---------------------------------------------------------

\ecvsection{Presentations and Editorial Activities}

  \ecvtitle{}{Editorial committee member}
  \ecvitem{}{ 
    \begin{ecvitemize}
      \item Associate Editor of IEEE Robotics and Automation Letters 2024.  
      \item Associate Editor of IEEE International Conference on Robotics and Automation (ICRA) 2023-2025.
      \item Program committee member of 4th International Conference on Robotics and Artificial Intelligence (ICRAI) 2018
      \item Program committee member of 2nd International Symposium on Artificial Intelligence and Robotics (ISAIR) 2017
      \item Program committee member of European Conference on Mobile Robotics (ECMR) 2011
    \end{ecvitemize}
  }

  \ecvtitle{}{Session chair or co-chair}
  \ecvitem{}{ 
    \begin{ecvitemize}
      \item IEEE Int.~Conf.~on Robotics and Automation (ICRA), Stoccolma (SWE), 2016
      \item IEEE/MTS Int.~Conf.~on Intelligent Robots and Systems (IROS), Daejeon (KOR), 2016
      \item IEEE/MTS OCEANS, Genova (IT), 2015
      \item IEEE/RSJ Int.~Conf.~on Intelligent Robots and Systems (IROS), Taipei (USA), 2010
      \item Intl.~Conf.~on Informatics in Control, Automation and Robotics (ICINCO), Anger (FR), 2007.
    \end{ecvitemize}
  }

  \ecvtitle{}{Reviewer}
  \ecvitem{}{ 
    Reviewer for several international journals including
    \begin{ecvitemize}
      \item IEEE~Transaction~on Robotics (T-RO)
      \item IEEE~Transaction~on Automation Science and Engineering (T-ASE)
      \item IEEE Robotics and Automation Letters (RA-L)
      \item Elsevier Robotics and Autonomous Systems (RAS)
      \item Springer Autonomous Robots (AURO)
      \item Springer Mechatronics
      \item Springer International Journal on~Control, Automation and Systems (IJCAS)
      \item MDPI Sensors
      \item MDPI Applied Sciences
    \end{ecvitemize}
  }
  \ecvitem{}{ 
    Reviewer for several international conferences including 
    \begin{ecvitemize}
      \item IEEE Int.~Conf.~on Robotics \& Automation (ICRA).
      \item IEEE/RSJ Int.~Conf.~on Intelligent Robots and Systems (IROS).
      \item Int.~Conf.~on Advanced Robotics (ICAR).
      \item European Conference on Mobile Robots (ECMR). 
      \item IEEE/MTS OCEANS. 
      \item AAAI Conference on Artificial Intelligence. 
      \item Spacial Cognition 2012.
      \item IEEE Intelligent Transportation Systems Society Conference Management System (ITSC).
      \item IEEE Control Systems Society Conference Management System (CDC).
      \item PID Conference.
    \end{ecvitemize}
  }

  \ecvtitle{}{Oral presentations}
  \ecvitem{}{ 
     \begin{ecvitemize}
      \item 2023: ICRA, London (UK). 
      \item 2021: SIMAI, Parma (IT).
      \item 2019: ICRA, Montreal (CA).
      \item 2018: IROS, Workshop, Madrid (ES).
      \item 2017: IROS, Vancouver (CA).
      \item 2016: ICRA, Stoccolma (SWE).
      \item 2016: IROS, Daejeon (KOR).
      \item 2016: IECON, Firenze (IT).
      \item 2015: OCEANS, Genova (IT).
      \item 2014: World Congr. of IFAC, Capetown (ZA).
      \item 2014: IAS e workshop AMRA, Padova (IT).
      \item 2013: ECMR, Barcelona (ES). 
      \item 2012: ICRA Workshop, St. Paul (USA). 
      \item 2012: ICINCO, Roma (IT). 
      \item 2011: ICRA, Shangai (CN). 
      \item 2010: IROS and Graphbot Workshop, Taipei (TW).
      \item 2009: ECMR, Dubrovnik (HR).
      \item 2009: IROS, St. Louis (USA).
      \item 2009: ICAR, Munich (DE).
      \item 2009: ICRA Workshop, Kobe (JP).
      \item 2008: CIRAS, Linz (JP).
      \item 2007: ECMR, Freiburg (DE).
      \item 2007: ICINCO, Angers (FR).
    \end{ecvitemize}
  }

% ---------------------------------------------------------
% TEACHING
% ---------------------------------------------------------

  \ecvsection{Teaching}

  \ecvtitle{2009 -- Present}{Robotica (Robotics)}
  \ecvitem{}{ Laurea Magistrale in Ingegneria Informatica (Master Degree in Computer Engineering), 
              Dipartimento di Ingegneria e Architettura, Universit\`a degli Studi di Parma,
              SSD ING/INF-05 ``Sistemi di elaborazione delle informazioni''
  }
  \ecvitem{}{
    \begin{ecvitemize}
      \item appointed regular professor from 2015 to present, contract professor from 2009 to 2015
      \item 6 CFU from 2011 (48 hours), 2/3 CFU module from 2009 to 2011
      \item previously in Laurea Specialistica in Ingegneria Informatica (2009-2010), 
            in Facolt\`a di Ingegneria and then Dipartimento di Ingegneria dell'Informazione
    \end{ecvitemize}
  }
  
  \ecvtitle{2023 -- Present}{Sistemi Operativi (Operating Systems)}
  \ecvitem{}{ Laurea in Ingegneria Informatica, Elettronica e delle Telecomunicazioni (Bachelor Degree in Computer, Electronics and Communication Engineering), 
              Dipartimento di Ingegneria e Architettura, Universit\`a degli Studi di Parma,
              SSD ING/INF-05 ``Sistemi di elaborazione delle informazioni''
  }
  \ecvitem{}{Role: appointed regular professor}
  
  \ecvtitle{2024 -- Present}{Fondamenti di Informatica (Introduction to Computer Science)}
  \ecvitem{}{ Laurea in Ingegneria Gestionale (Bachelor Degree in Management Engineering), 
              Dipartimento di Ingegneria dei Sistemi e delle Tecnologie Industriali, Universit\`a degli Studi di Parma,
              SSD ING/INF-05 ``Sistemi di elaborazione delle informazioni''
  }
  \ecvitem{}{Role: co-appointed regular professor}
  
  \ecvtitle{2022 -- 2023}{Sistemi Operativi e in Tempo Reale (Operating and Real-Time Systems)}
  \ecvitem{}{ Laurea Magistrale in Ingegneria Informatica (Master Degree in Computer Engineering), 
              Dipartimento di Ingegneria e Architettura, Universit\`a degli Studi di Parma,
              SSD ING/INF-05 ``Sistemi di elaborazione delle informazioni''
  }
  \ecvitem{}{Role: co-appointed regular professor}

  \ecvtitle{2009 -- 2010}{Sistemi Operativi A (Operating Systems)}
  \ecvitem{}{ Laurea in Ingegneria Informatica con didattica a distanza (Bachelor Degree in Computer Engineering), 
              Universit\`a degli Studi di Parma,
              SSD ING/INF-05 ``Sistemi di elaborazione delle informazioni''
  }
  \ecvitem{}{Role: contract professor}

  \ecvtitle{2009--2015 and 2017--2020}{Sistemi Operativi e in Tempo Reale (Operating and Real-Time Systems)}
  \ecvitem{}{ Laurea Magistrale in Ingegneria Informatica (Master Degree in Computer Engineering), 
              Dipartimento di Ingegneria e Architettura, Universit\`a degli Studi di Parma,
              SSD ING/INF-05 ``Sistemi di elaborazione delle informazioni''
  }
  \ecvitem{}{Role: laboratory practicals and exercises }

  \ecvtitle{2015--2017}{Sistemi Operativi A (Operating and Real-Time Systems)}
  \ecvitem{}{ Laurea in Ingegneria Informatica, Elettronica e delle Telecomunicazioni (Master Degree in Computer, Electronic and Communication Engineering), 
              Dipartimento di Ingegneria e Architettura, Universit\`a degli Studi di Parma,
              SSD ING/INF-05 ``Sistemi di elaborazione delle informazioni''
  }
  \ecvitem{}{Role: laboratory practicals and exercises }

  \ecvitem{2006--2009}{Robotica (Robotics)}
  \ecvitem{}{ Laurea Specialistica in Ingegneria Informatica (Master Degree in Computer Engineering), 
              Universit\`a degli Studi di Parma,
              SSD ING/INF-05 ``Sistemi di elaborazione delle informazioni''
  }
  \ecvitem{}{Role: laboratory practicals and exercises }

  \ecvtitle{March-September 2005}{Controlli Automatici A (Automatic Control)}
  \ecvitem{}{ Lauree in Ingegneria Informatica, Elettronica e delle Telecomunicazioni (Master Degrees in Computer, Electronic and Communication Engineering), 
              Facolt\`a di Ingegneria, Universit\`a degli Studi di Parma,
              SSD ING/INF-04
  }
  \ecvitem{}{Role: tutor}

% ---------------------------------------------------------
% OTHER EXPERIENCES
% ---------------------------------------------------------

  \ecvsection{Other Experiences and Qualifications}

\ecvitem{2018-2023}{Referee of PhD Dissertations}
  \ecvitem{}{(Revisore di tesi di dottorato di candidati dell'Universit\`a degli Studi di Padova e dell'Universit\`a Politecnica delle Marche)}

  \ecvitem{2018}{Examiner for PhD Dissertation Defense}
  \ecvitem{}{(Membro interno della Commissione Esaminatrice del Dottorato in Tecnologie dell'Informazione, ciclo XXXI, Universit\`a degli Studi di Parma)}
  \ecvitem{}{Role: commitee member}

  \ecvitem{2018--2019}{Professional Engineering Examination Committee}
  \ecvitem{}{(Commissione per Esame di Stato per l'abilitazione alla Professione di Ingegnere di Parma)}
  \ecvitem{}{Role: expert member}

  \ecvitem{2006}{Engineering Professional Qualification}
  \ecvitem{}{(Abilitato alla professione di Ingegnere, settore dell'Informazione, sezione A)}

% ---------------------------------------------------------
% PERSONAL SKILLS
% ---------------------------------------------------------
  
  \ecvsection{Personal skills}
  \ecvmothertongue{Italian}
  \ecvlanguageheader
  \ecvlanguage{English}{C1}{C2}{B2}{C1}{C2}
  \ecvlanguage{French}{A2}{A2}{A2}{A2}{A2}
%  \ecvlanguagecertificate{Diplôme d'études en langue française (DELF) B1}
  \ecvlanguagefooter
   
%  \ecvblueitem{Communication skills}{
%  \begin{ecvitemize}
%    \item team work: I have worked in various types of teams from research teams to national league hockey. For 2 years I coached my university hockey team
%    \item mediating skills: I work on the borders between young people, youth trainers, youth policy and researchers, for example running a 3 day workshop at CoE Symposium ``Youth Actor of Social Change'', and my continued work on youth training programmes 
%    \item intercultural skills: I am experienced at working in a European dimension such as being a rapporteur at the CoE Budapest ``youth against violence seminar'' and working with refugees.
%  \end{ecvitemize}
%  }
  
  \ecvblueitem{Organisational / managerial skills}{
  \begin{ecvitemize}
    \item I have been advisor or co-advisor of PhD students and supervisor of more than 40 master and bachelor theses
    \item I have been principal investigator of research contracts and regional projects as well as supervisor of work-package in national projects
    \item I have been the leader of the student team taking part to Sick Robot Day international robotic competitions in 2010, 2012 and 2014, winning the later two editions
  \end{ecvitemize}
  }

  \ecvdigitalcompetence{\ecvProficient}{\ecvProficient}{\ecvProficient}{\ecvProficient}{\ecvIndependent}
  
  \ecvblueitem{Computer skills}{
  \begin{ecvitemize}
    \item Programming languages: C, C++ (including C++-11/14), Matlab/Octave, Python, Java, web scripting languages (basic)
    \item Programming design: experience with object and generic programming, design patterns
    \item Development tools: CMake, Git, SVN, Subversion (SVN), MS VisualStudio, several IDEs
    \item Operating systems: Linux (daily experience), Windows
    \item Specific tools: Robotic Operating System (ROS), several robotic software tools
    \item Robots: experience with Universal Robots manipulators (UR10e), Fanuc Crx10ial, Comau Smart Six, mobile robot platforms
  \end{ecvitemize}
  }

% ---------------------------------------------------------
% PUBBLICACTIONS
% ---------------------------------------------------------

%  \pagebreak

  \ecvsection{Publications}

%  \bibliographystyle{unsrt_dlr}
%  \nobibliography{dlr_pubblications}
  
  \ecvblueitem{Journals}{}
      \BIBCITEM{monica2024tvcg}
      \BIBCITEM{penzotti2023compag}      
      \BIBCITEM{fontana2023sensors}
      \BIBCITEM{aleotti2021ap}
      %\BIBCITEM{consolini2020pr}
      \BIBCITEM{lodirizzini2019ral}
      \BIBCITEM{galasso2019rcim}
      \BIBCITEM{lodirizzini2018pr}
      \BIBCITEM{simetti2018joe}
      \BIBCITEM{kallasi2017ras}
      \BIBCITEM{lodirizzini2017caee}
      \BIBCITEM{casalino2016mtsj}
      \BIBCITEM{kallasi2016ral}
      \BIBCITEM{lodirizzini2015ijars}
      \BIBCITEM{aleotti2014jirs}
      \BIBCITEM{cigolini2014jamris}
      \BIBCITEM{lodirizzini2009ras}
  \ecvblueitem{Conference procedings}{}
      % 
      \BIBCITEM{monica2024etfa}
      %
      \BIBCITEM{penzotti2023ecpa}
      %
      \BIBCITEM{khan2022irc}
      \BIBCITEM{lodirizzini2022raad}
      %
      \BIBCITEM{fontana2021ecmr}
      \BIBCITEM{fontana2021iccp}
      \BIBCITEM{khan2021iccp}
      \BIBCITEM{lodirizzini2021simai}
      %
      \BIBCITEM{monica2020irc}
      \BIBCITEM{chiaravalli2020etfa}
      \BIBCITEM{amoretti2020smartsys}
      %
      \BIBCITEM{lodirizzini2018irosws}
      %
      \BIBCITEM{lodirizzini2017iros}
      %
      \BIBCITEM{kallasi2016iros}
      \BIBCITEM{oleari2016iecon}
      %
      \BIBCITEM{kallasi2015oceans}
      \BIBCITEM{oleari2015oceans}
      %
      \BIBCITEM{aleotti2014iros}
      \BIBCITEM{oleari2014ifac}
      \BIBCITEM{lodirizzini2014ias}
      \BIBCITEM{kallasi2014iarp}
      %
      \BIBCITEM{lodirizzini2013ecmr}
      \BIBCITEM{valeriani2013acra}
      \BIBCITEM{oleari2013iccp}
      %
      \BIBCITEM{aleotti2012icra}
      \BIBCITEM{lodirizzini2012icinco}
      \BIBCITEM{calo2012icinco}
      %
      \BIBCITEM{lodirizzini2011icra}
      %
      \BIBCITEM{lodirizzini2010iros}
      \BIBCITEM{argenti2010isr}
      %
      \BIBCITEM{lodirizzini2009iros}
      \BIBCITEM{lodirizzini2009ecmr}
      \BIBCITEM{lodirizzini2009icar}
      %
      \BIBCITEM{grisetti2008icra}
      \BIBCITEM{lodirizzini2008ciras}
      %
      \BIBCITEM{lodirizzini2007ecmr}
      \BIBCITEM{lodirizzini2007icinco}
  \ecvblueitem{Workshops}{}
      \BIBCITEM{fontana2023icraws}
      %
      \BIBCITEM{lodirizzini2021simai}
      %
      \BIBCITEM{kallasi2014amra}
      %
      \BIBCITEM{lodirizzini2012icraworkshop}
      %
      \BIBCITEM{lodirizzini2010graphbot}
      %
      \BIBCITEM{lodirizzini2009icraworkshop}
      %
      \BIBCITEM{cerri2007ccmvs}
      %
   \ecvblueitem{Book chapters}{}
      \BIBCITEM{lodirizzini2008improved}
   \ecvblueitem{PhD Dissertation}{}
      \BIBCITEM{lodirizzini2009thesis}
  
  \end{europasscv}

\end{document}
